% Options for packages loaded elsewhere
\PassOptionsToPackage{unicode}{hyperref}
\PassOptionsToPackage{hyphens}{url}
%
\documentclass[
  11pt,
  ignorenonframetext,
]{beamer}
\usepackage{pgfpages}
\setbeamertemplate{caption}[numbered]
\setbeamertemplate{caption label separator}{: }
\setbeamercolor{caption name}{fg=normal text.fg}
\beamertemplatenavigationsymbolsempty
% Prevent slide breaks in the middle of a paragraph
\widowpenalties 1 10000
\raggedbottom
\setbeamertemplate{part page}{
  \centering
  \begin{beamercolorbox}[sep=16pt,center]{part title}
    \usebeamerfont{part title}\insertpart\par
  \end{beamercolorbox}
}
\setbeamertemplate{section page}{
  \centering
  \begin{beamercolorbox}[sep=12pt,center]{part title}
    \usebeamerfont{section title}\insertsection\par
  \end{beamercolorbox}
}
\setbeamertemplate{subsection page}{
  \centering
  \begin{beamercolorbox}[sep=8pt,center]{part title}
    \usebeamerfont{subsection title}\insertsubsection\par
  \end{beamercolorbox}
}
\AtBeginPart{
  \frame{\partpage}
}
\AtBeginSection{
  \ifbibliography
  \else
    \frame{\sectionpage}
  \fi
}
\AtBeginSubsection{
  \frame{\subsectionpage}
}
\usepackage{amsmath,amssymb}
\usepackage{lmodern}
\usepackage{iftex}
\ifPDFTeX
  \usepackage[T1]{fontenc}
  \usepackage[utf8]{inputenc}
  \usepackage{textcomp} % provide euro and other symbols
\else % if luatex or xetex
  \usepackage{unicode-math}
  \defaultfontfeatures{Scale=MatchLowercase}
  \defaultfontfeatures[\rmfamily]{Ligatures=TeX,Scale=1}
\fi
\usetheme[]{metropolis}
% Use upquote if available, for straight quotes in verbatim environments
\IfFileExists{upquote.sty}{\usepackage{upquote}}{}
\IfFileExists{microtype.sty}{% use microtype if available
  \usepackage[]{microtype}
  \UseMicrotypeSet[protrusion]{basicmath} % disable protrusion for tt fonts
}{}
\makeatletter
\@ifundefined{KOMAClassName}{% if non-KOMA class
  \IfFileExists{parskip.sty}{%
    \usepackage{parskip}
  }{% else
    \setlength{\parindent}{0pt}
    \setlength{\parskip}{6pt plus 2pt minus 1pt}}
}{% if KOMA class
  \KOMAoptions{parskip=half}}
\makeatother
\usepackage{xcolor}
\newif\ifbibliography
\setlength{\emergencystretch}{3em} % prevent overfull lines
\providecommand{\tightlist}{%
  \setlength{\itemsep}{0pt}\setlength{\parskip}{0pt}}
\setcounter{secnumdepth}{-\maxdimen} % remove section numbering
\ifLuaTeX
  \usepackage{selnolig}  % disable illegal ligatures
\fi
\IfFileExists{bookmark.sty}{\usepackage{bookmark}}{\usepackage{hyperref}}
\IfFileExists{xurl.sty}{\usepackage{xurl}}{} % add URL line breaks if available
\urlstyle{same} % disable monospaced font for URLs
\hypersetup{
  pdftitle={Planteamiento y diseño de la investigación},
  pdfauthor={Gerardo Martín},
  hidelinks,
  pdfcreator={LaTeX via pandoc}}

\title{Planteamiento y diseño de la investigación}
\subtitle{Preguntas e hipótesis}
\author{Gerardo Martín}
\date{2022-06-29}

\begin{document}
\frame{\titlepage}

\hypertarget{preguntas}{%
\section{Preguntas}\label{preguntas}}

\begin{frame}{La pregunta de investigación}
\protect\hypertarget{la-pregunta-de-investigaciuxf3n}{}
\begin{enumerate}
\item
  Nace de la curiosidad
\item
  Aquello que tu investigación busca responder
\item
  No hay una regla sobre la naturaleza de la pregunta
\end{enumerate}
\end{frame}

\begin{frame}{Ejemplos}
\protect\hypertarget{ejemplos}{}
\begin{itemize}
\item
  ¿Cómo sucede X?
\item
  ¿Qué es más importante para Z, X ó Y?
\item
  ¿Cuándo comienza tal fenómeno?
\item
  ¿Quién es responsable de X, Y, Z?
\item
  ¿Dónde ocurre X ó Y fenómeno?
\item
  ¿Por qué observamos ABC?
\end{itemize}
\end{frame}

\begin{frame}{Características de las preguntas}
\protect\hypertarget{caracteruxedsticas-de-las-preguntas}{}
\begin{itemize}
\item
  Claras y fáciles de comprender
\item
  Esperdíficas
\item
  Respondibles
\item
  Relevantes para el área de estudio
\end{itemize}
\end{frame}

\begin{frame}{Hipótesis}
\protect\hypertarget{hipuxf3tesis}{}
\begin{itemize}
\item
  Aseveración, proposición

  \begin{itemize}
  \item
    Relacionada con el funcionamiento del sistema de estudio
  \item
    Respuesta posible a la pregunta de investigación
  \end{itemize}
\item
  Desarrolladas a partir de:

  \begin{itemize}
  \item
    Observaciones informales
  \item
    Estudio sistemático del fenómeno
  \end{itemize}
\end{itemize}
\end{frame}

\begin{frame}{Características de las hipótesis}
\protect\hypertarget{caracteruxedsticas-de-las-hipuxf3tesis}{}
\begin{itemize}
\item
  Claras y fáciles de comprender
\item
  Específicas
\item
  Respondibles
\item
  Relevantes para el área de estudio
\end{itemize}
\end{frame}

\begin{frame}{Para qué?}
\protect\hypertarget{para-quuxe9}{}
\begin{itemize}
\item
  Estimular el proceso de investigación
\item
  Gestionarlo

  \begin{itemize}
  \item
    Enfoque
  \item
    Priorizar
  \end{itemize}
\item
  Identificar los datos necesarios para responder las preguntas ó probar
  las hipótesis
\end{itemize}
\end{frame}

\hypertarget{actividad}{%
\section{Actividad}\label{actividad}}

\begin{frame}{La conquista de los sapos}
\protect\hypertarget{la-conquista-de-los-sapos}{}
\url{https://www.youtube.com/watch?v=P-3LevPwLz4}
\end{frame}

\begin{frame}{Preguntas}
\protect\hypertarget{preguntas-1}{}
\begin{itemize}
\item
  ¿Fue posible comprender las preguntas de tu compañerx?
\item
  ¿Las hipótesis están relacionadas con las preguntas de investigación?
\item
  ¿Las preguntas/hipótesis sirven para identificar los datos que se
  necesita colectar para responder o probarlas?
\end{itemize}
\end{frame}

\end{document}
