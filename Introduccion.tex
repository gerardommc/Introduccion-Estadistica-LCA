% Options for packages loaded elsewhere
\PassOptionsToPackage{unicode}{hyperref}
\PassOptionsToPackage{hyphens}{url}
%
\documentclass[
  11pt,
  ignorenonframetext,
]{beamer}
\usepackage{pgfpages}
\setbeamertemplate{caption}[numbered]
\setbeamertemplate{caption label separator}{: }
\setbeamercolor{caption name}{fg=normal text.fg}
\beamertemplatenavigationsymbolsempty
% Prevent slide breaks in the middle of a paragraph
\widowpenalties 1 10000
\raggedbottom
\setbeamertemplate{part page}{
  \centering
  \begin{beamercolorbox}[sep=16pt,center]{part title}
    \usebeamerfont{part title}\insertpart\par
  \end{beamercolorbox}
}
\setbeamertemplate{section page}{
  \centering
  \begin{beamercolorbox}[sep=12pt,center]{part title}
    \usebeamerfont{section title}\insertsection\par
  \end{beamercolorbox}
}
\setbeamertemplate{subsection page}{
  \centering
  \begin{beamercolorbox}[sep=8pt,center]{part title}
    \usebeamerfont{subsection title}\insertsubsection\par
  \end{beamercolorbox}
}
\AtBeginPart{
  \frame{\partpage}
}
\AtBeginSection{
  \ifbibliography
  \else
    \frame{\sectionpage}
  \fi
}
\AtBeginSubsection{
  \frame{\subsectionpage}
}
\usepackage{amsmath,amssymb}
\usepackage{lmodern}
\usepackage{iftex}
\ifPDFTeX
  \usepackage[T1]{fontenc}
  \usepackage[utf8]{inputenc}
  \usepackage{textcomp} % provide euro and other symbols
\else % if luatex or xetex
  \usepackage{unicode-math}
  \defaultfontfeatures{Scale=MatchLowercase}
  \defaultfontfeatures[\rmfamily]{Ligatures=TeX,Scale=1}
\fi
\usetheme[]{metropolis}
% Use upquote if available, for straight quotes in verbatim environments
\IfFileExists{upquote.sty}{\usepackage{upquote}}{}
\IfFileExists{microtype.sty}{% use microtype if available
  \usepackage[]{microtype}
  \UseMicrotypeSet[protrusion]{basicmath} % disable protrusion for tt fonts
}{}
\makeatletter
\@ifundefined{KOMAClassName}{% if non-KOMA class
  \IfFileExists{parskip.sty}{%
    \usepackage{parskip}
  }{% else
    \setlength{\parindent}{0pt}
    \setlength{\parskip}{6pt plus 2pt minus 1pt}}
}{% if KOMA class
  \KOMAoptions{parskip=half}}
\makeatother
\usepackage{xcolor}
\newif\ifbibliography
\setlength{\emergencystretch}{3em} % prevent overfull lines
\providecommand{\tightlist}{%
  \setlength{\itemsep}{0pt}\setlength{\parskip}{0pt}}
\setcounter{secnumdepth}{-\maxdimen} % remove section numbering
\ifLuaTeX
  \usepackage{selnolig}  % disable illegal ligatures
\fi
\IfFileExists{bookmark.sty}{\usepackage{bookmark}}{\usepackage{hyperref}}
\IfFileExists{xurl.sty}{\usepackage{xurl}}{} % add URL line breaks if available
\urlstyle{same} % disable monospaced font for URLs
\hypersetup{
  pdftitle={Conceptos básicos de estadística},
  pdfauthor={Gerardo Martín},
  hidelinks,
  pdfcreator={LaTeX via pandoc}}

\title{Conceptos básicos de estadística}
\subtitle{¿Qué es y por qué la necesitamos}
\author{Gerardo Martín}
\date{2022-06-29}

\begin{document}
\frame{\titlepage}

\hypertarget{quuxe9-es}{%
\section{¿Qué es?}\label{quuxe9-es}}

\begin{frame}{Definiciones}
\protect\hypertarget{definiciones}{}
\begin{itemize}
\tightlist
\item
  Colectar
\item
  Organizar
\item
  Resumir
\item
  Analizar
\end{itemize}

Pregunta \(\rightarrow\) \textbf{Información} \(\rightarrow\) Respuesta
\(\rightarrow\) Certidumbre
\end{frame}

\begin{frame}{¿Qué es la \emph{información}?}
\protect\hypertarget{quuxe9-es-la-informaciuxf3n}{}
\begin{itemize}
\item
  Hechos, propuestas, mediciones

  \begin{itemize}
  \tightlist
  \item
    Descripciones del mundo
  \item
    Colectar, organizar, resumir, analizar
  \end{itemize}
\item
  Propósito: tomar decisiones
\end{itemize}

\textbf{Datos}
\end{frame}

\begin{frame}{¿Cómo pueden ser las descripciones del mundo?}
\protect\hypertarget{cuxf3mo-pueden-ser-las-descripciones-del-mundo}{}
\begin{itemize}
\item
  Numéricas

  \begin{itemize}
  \tightlist
  \item
    Estatura
  \item
    Cantidad de personas
  \item
    Temperatura
  \end{itemize}
\item
  Categóricas

  \begin{itemize}
  \tightlist
  \item
    Sexo o Género
  \item
    País de origen
  \item
    Color
  \end{itemize}
\end{itemize}
\end{frame}

\begin{frame}{Otras características de los datos}
\protect\hypertarget{otras-caracteruxedsticas-de-los-datos}{}
\begin{itemize}
\item
  Pueden tomar muchos valores distintos
\item
  Entre uds.:

  \begin{itemize}
  \tightlist
  \item
    ¿cuántas estaturas diferentes hay?
  \item
    ¿cuántas identidades diferentes hay?
  \item
    ¿cuántos intereses profesionales hay?
  \end{itemize}
\end{itemize}
\end{frame}

\begin{frame}{Otras características de los datos}
\protect\hypertarget{otras-caracteruxedsticas-de-los-datos-1}{}
\begin{itemize}
\item
  Sobre c/u de uds.:

  \begin{itemize}
  \tightlist
  \item
    ¿usan la misma combinación de ropa todos los días?
  \item
    ¿comen lo mismo todos los días?
  \item
    ¿llegan exactamente a la misma hora?
  \item
    ¿duermen la misma cantidad de horas todos los días?
  \end{itemize}
\end{itemize}
\end{frame}

\begin{frame}{Sobre los datos}
\protect\hypertarget{sobre-los-datos}{}
\begin{enumerate}
\item
  Mediciones o descripciones
\item
  Son variables

  \begin{itemize}
  \tightlist
  \item
    Grupo
  \item
    Individuo
  \end{itemize}
\end{enumerate}

¿Qué origina la variación dentro de cada grupo e individuo?
\end{frame}

\begin{frame}{``\,''}
\protect\hypertarget{section}{}
La estadística estudia la variabilidad del mundo y sus causas. Para ello
utiliza datos colectados sistemáticamente, organizándolos y
analizándolos de acuerdo con procedimientos matemáticos.
\end{frame}

\end{document}
