% Options for packages loaded elsewhere
\PassOptionsToPackage{unicode}{hyperref}
\PassOptionsToPackage{hyphens}{url}
%
\documentclass[
  11pt,
  ignorenonframetext,
]{beamer}
\usepackage{pgfpages}
\setbeamertemplate{caption}[numbered]
\setbeamertemplate{caption label separator}{: }
\setbeamercolor{caption name}{fg=normal text.fg}
\beamertemplatenavigationsymbolsempty
% Prevent slide breaks in the middle of a paragraph
\widowpenalties 1 10000
\raggedbottom
\setbeamertemplate{part page}{
  \centering
  \begin{beamercolorbox}[sep=16pt,center]{part title}
    \usebeamerfont{part title}\insertpart\par
  \end{beamercolorbox}
}
\setbeamertemplate{section page}{
  \centering
  \begin{beamercolorbox}[sep=12pt,center]{part title}
    \usebeamerfont{section title}\insertsection\par
  \end{beamercolorbox}
}
\setbeamertemplate{subsection page}{
  \centering
  \begin{beamercolorbox}[sep=8pt,center]{part title}
    \usebeamerfont{subsection title}\insertsubsection\par
  \end{beamercolorbox}
}
\AtBeginPart{
  \frame{\partpage}
}
\AtBeginSection{
  \ifbibliography
  \else
    \frame{\sectionpage}
  \fi
}
\AtBeginSubsection{
  \frame{\subsectionpage}
}
\usepackage{amsmath,amssymb}
\usepackage{lmodern}
\usepackage{iftex}
\ifPDFTeX
  \usepackage[T1]{fontenc}
  \usepackage[utf8]{inputenc}
  \usepackage{textcomp} % provide euro and other symbols
\else % if luatex or xetex
  \usepackage{unicode-math}
  \defaultfontfeatures{Scale=MatchLowercase}
  \defaultfontfeatures[\rmfamily]{Ligatures=TeX,Scale=1}
\fi
\usetheme[]{metropolis}
% Use upquote if available, for straight quotes in verbatim environments
\IfFileExists{upquote.sty}{\usepackage{upquote}}{}
\IfFileExists{microtype.sty}{% use microtype if available
  \usepackage[]{microtype}
  \UseMicrotypeSet[protrusion]{basicmath} % disable protrusion for tt fonts
}{}
\makeatletter
\@ifundefined{KOMAClassName}{% if non-KOMA class
  \IfFileExists{parskip.sty}{%
    \usepackage{parskip}
  }{% else
    \setlength{\parindent}{0pt}
    \setlength{\parskip}{6pt plus 2pt minus 1pt}}
}{% if KOMA class
  \KOMAoptions{parskip=half}}
\makeatother
\usepackage{xcolor}
\newif\ifbibliography
\usepackage{graphicx}
\makeatletter
\def\maxwidth{\ifdim\Gin@nat@width>\linewidth\linewidth\else\Gin@nat@width\fi}
\def\maxheight{\ifdim\Gin@nat@height>\textheight\textheight\else\Gin@nat@height\fi}
\makeatother
% Scale images if necessary, so that they will not overflow the page
% margins by default, and it is still possible to overwrite the defaults
% using explicit options in \includegraphics[width, height, ...]{}
\setkeys{Gin}{width=\maxwidth,height=\maxheight,keepaspectratio}
% Set default figure placement to htbp
\makeatletter
\def\fps@figure{htbp}
\makeatother
\setlength{\emergencystretch}{3em} % prevent overfull lines
\providecommand{\tightlist}{%
  \setlength{\itemsep}{0pt}\setlength{\parskip}{0pt}}
\setcounter{secnumdepth}{-\maxdimen} % remove section numbering
\ifLuaTeX
  \usepackage{selnolig}  % disable illegal ligatures
\fi
\IfFileExists{bookmark.sty}{\usepackage{bookmark}}{\usepackage{hyperref}}
\IfFileExists{xurl.sty}{\usepackage{xurl}}{} % add URL line breaks if available
\urlstyle{same} % disable monospaced font for URLs
\hypersetup{
  pdftitle={Conceptos básicos de estadística},
  pdfauthor={Gerardo Martín},
  hidelinks,
  pdfcreator={LaTeX via pandoc}}

\title{Conceptos básicos de estadística}
\subtitle{Consideraciones en la selección de la población}
\author{Gerardo Martín}
\date{2022-06-29}

\begin{document}
\frame{\titlepage}

\begin{frame}{Generalizar}
\protect\hypertarget{generalizar}{}
Muestra \(\rightarrow\) Población

Particular \(\rightarrow\) General
\end{frame}

\begin{frame}{Condición}
\protect\hypertarget{condiciuxf3n}{}
Para generalizar a partir de muestra necesitamos que:

Muestra \(\approx\) Población
\end{frame}

\begin{frame}{¿Qué pasa si no se cumple?}
\protect\hypertarget{quuxe9-pasa-si-no-se-cumple}{}
\includegraphics{Figuras-poblacion/Sesgo.png}
\end{frame}

\begin{frame}{¿Qué pasa si no se cumple?}
\protect\hypertarget{quuxe9-pasa-si-no-se-cumple-1}{}
\begin{enumerate}
\item
  Descripciones no reflejan realidad
\item
  Riesgoso hacer recomendaciones basadas en evidencia sesgada
\item
  Manejo ineficiente

  \begin{itemize}
  \tightlist
  \item
    Consecuencias políticas
  \end{itemize}
\end{enumerate}
\end{frame}

\begin{frame}{Ejemplo}
\protect\hypertarget{ejemplo}{}
\includegraphics{Figuras-poblacion/Sesgo-1.png}
\end{frame}

\begin{frame}{Ejemplo}
\protect\hypertarget{ejemplo-1}{}
En la muestra del polígono gris

\begin{itemize}
\tightlist
\item
  ¿Cuál es el color más común?
\end{itemize}

En la población

\begin{itemize}
\tightlist
\item
  ¿Cuál es el color más común?
\end{itemize}
\end{frame}

\begin{frame}{Soluciones}
\protect\hypertarget{soluciones}{}
\includegraphics{Figuras-poblacion/Sesgo-2.png}
\end{frame}

\begin{frame}{Soluciones}
\protect\hypertarget{soluciones-1}{}
\begin{itemize}
\item
  Muestra más grande
\item
  Muestra más aleatoria
\item
  Diseño de muestreo/experimental
\end{itemize}
\end{frame}

\begin{frame}{Estrategias para obtener datos}
\protect\hypertarget{estrategias-para-obtener-datos}{}
\begin{enumerate}
\item
  Experimentos

  \begin{itemize}
  \tightlist
  \item
    Diseñar procedimiento para replicar sistema o componentes
  \end{itemize}
\item
  Observaciones

  \begin{itemize}
  \tightlist
  \item
    Hacer mediciones directamente en campo
  \end{itemize}
\end{enumerate}
\end{frame}

\begin{frame}{Ejemplos}
\protect\hypertarget{ejemplos}{}
\begin{itemize}
\tightlist
\item
  Experimentos
\end{itemize}

\href{https://www.youtube.com/watch?v=BuIYh17daSs}{Evolución del cauce
de un río}
\end{frame}

\begin{frame}{Ejemplos}
\protect\hypertarget{ejemplos-1}{}
\begin{itemize}
\tightlist
\item
  Observaciones
\end{itemize}

\begin{enumerate}
\item
  \href{https://www.sciencedirect.com/science/article/abs/pii/S0013935120314961}{Efectos
  del \emph{Greenness} en la natalidad}
\item
  \href{https://onlinelibrary.wiley.com/doi/abs/10.1002/ajim.22946?casa_token=YsxyobqagVoAAAAA:wQw6Ys7vyJwmpNFknOv29Y88asNazOdFFxWnQ-gvo87DQ9sjU1wbOuPml275U9NDWdcze3fsDCGO0Vw}{Impacto
  del estrés por calor en el riesgo de lesiones ocupacionales}
\end{enumerate}
\end{frame}

\begin{frame}{Estrategias observacionales}
\protect\hypertarget{estrategias-observacionales}{}
\begin{enumerate}
\item
  Colectar muestras en campo

  \begin{itemize}
  \tightlist
  \item
    Muestras de sedimento de río
  \item
    Entrevistas a actores clave
  \end{itemize}
\item
  Consulta de registros gubernamentales
\item
  Encuestas

  \begin{itemize}
  \tightlist
  \item
    Visitas a comunidades
  \item
    Virtuales
  \end{itemize}
\item
  Ciencia ciudadana

  \begin{itemize}
  \tightlist
  \item
    \href{https://naturalista.mx}{Naturalista MX}
  \end{itemize}
\end{enumerate}
\end{frame}

\end{document}
