% Options for packages loaded elsewhere
\PassOptionsToPackage{unicode}{hyperref}
\PassOptionsToPackage{hyphens}{url}
%
\documentclass[
  11pt,
  ignorenonframetext,
]{beamer}
\usepackage{pgfpages}
\setbeamertemplate{caption}[numbered]
\setbeamertemplate{caption label separator}{: }
\setbeamercolor{caption name}{fg=normal text.fg}
\beamertemplatenavigationsymbolsempty
% Prevent slide breaks in the middle of a paragraph
\widowpenalties 1 10000
\raggedbottom
\setbeamertemplate{part page}{
  \centering
  \begin{beamercolorbox}[sep=16pt,center]{part title}
    \usebeamerfont{part title}\insertpart\par
  \end{beamercolorbox}
}
\setbeamertemplate{section page}{
  \centering
  \begin{beamercolorbox}[sep=12pt,center]{part title}
    \usebeamerfont{section title}\insertsection\par
  \end{beamercolorbox}
}
\setbeamertemplate{subsection page}{
  \centering
  \begin{beamercolorbox}[sep=8pt,center]{part title}
    \usebeamerfont{subsection title}\insertsubsection\par
  \end{beamercolorbox}
}
\AtBeginPart{
  \frame{\partpage}
}
\AtBeginSection{
  \ifbibliography
  \else
    \frame{\sectionpage}
  \fi
}
\AtBeginSubsection{
  \frame{\subsectionpage}
}
\usepackage{amsmath,amssymb}
\usepackage{lmodern}
\usepackage{iftex}
\ifPDFTeX
  \usepackage[T1]{fontenc}
  \usepackage[utf8]{inputenc}
  \usepackage{textcomp} % provide euro and other symbols
\else % if luatex or xetex
  \usepackage{unicode-math}
  \defaultfontfeatures{Scale=MatchLowercase}
  \defaultfontfeatures[\rmfamily]{Ligatures=TeX,Scale=1}
\fi
\usetheme[]{metropolis}
% Use upquote if available, for straight quotes in verbatim environments
\IfFileExists{upquote.sty}{\usepackage{upquote}}{}
\IfFileExists{microtype.sty}{% use microtype if available
  \usepackage[]{microtype}
  \UseMicrotypeSet[protrusion]{basicmath} % disable protrusion for tt fonts
}{}
\makeatletter
\@ifundefined{KOMAClassName}{% if non-KOMA class
  \IfFileExists{parskip.sty}{%
    \usepackage{parskip}
  }{% else
    \setlength{\parindent}{0pt}
    \setlength{\parskip}{6pt plus 2pt minus 1pt}}
}{% if KOMA class
  \KOMAoptions{parskip=half}}
\makeatother
\usepackage{xcolor}
\newif\ifbibliography
\setlength{\emergencystretch}{3em} % prevent overfull lines
\providecommand{\tightlist}{%
  \setlength{\itemsep}{0pt}\setlength{\parskip}{0pt}}
\setcounter{secnumdepth}{-\maxdimen} % remove section numbering
\ifLuaTeX
  \usepackage{selnolig}  % disable illegal ligatures
\fi
\IfFileExists{bookmark.sty}{\usepackage{bookmark}}{\usepackage{hyperref}}
\IfFileExists{xurl.sty}{\usepackage{xurl}}{} % add URL line breaks if available
\urlstyle{same} % disable monospaced font for URLs
\hypersetup{
  pdftitle={Planteamiento y diseño de la investigación},
  pdfauthor={Gerardo Martín},
  hidelinks,
  pdfcreator={LaTeX via pandoc}}

\title{Planteamiento y diseño de la investigación}
\subtitle{Tipos de estudios}
\author{Gerardo Martín}
\date{2022-06-29}

\begin{document}
\frame{\titlepage}

\begin{frame}{Clasificación de estudios}
\protect\hypertarget{clasificaciuxf3n-de-estudios}{}
\begin{itemize}
\item
  Observacionales
\item
  Experimentales
\end{itemize}
\end{frame}

\hypertarget{observacionales}{%
\section{Observacionales}\label{observacionales}}

\begin{frame}{Características}
\protect\hypertarget{caracteruxedsticas}{}
\begin{itemize}
\item
  Observación del estado natural
\item
  No hay manipulación
\item
  Necesario diseñar cuidadosamente

  \begin{itemize}
  \tightlist
  \item
    Evitar muestra sesgada
  \end{itemize}
\end{itemize}
\end{frame}

\begin{frame}{Ejemplos}
\protect\hypertarget{ejemplos}{}
\emph{Un equipo de científicos acude regularmente a sitios alrededor de
la península de Yucatán para medir el efecto del cambio climático sobre
la concentración de nitrógeno y fósforo en el suelo}
\end{frame}

\begin{frame}{Ejemplos}
\protect\hypertarget{ejemplos-1}{}
\emph{En los inicios de la pandemia hubo un brote de COVID-19. El buque
fue cuarentenado en Japón y terminado el brote un equipo de
epidemiólogos trató de estimar cómo cambiaba la letalidad del virus con
la edad}
\end{frame}

\begin{frame}{Ejemplos}
\protect\hypertarget{ejemplos-2}{}
\emph{El borrego cimarrón es muy susceptible a la neumonía. Kezia
Manlove detectó por medio de muestras de individuos que cuando el brote
comienza en una población sus números comienzan a disminuir.}
\end{frame}

\begin{frame}{Debilidades y fortalezas}
\protect\hypertarget{debilidades-y-fortalezas}{}
\begin{itemize}
\item
  Conclusiones poco confiables y generalizables
\item
  Difícil determinar causas
\item
  Más realistas
\end{itemize}
\end{frame}

\hypertarget{experimentales}{%
\section{Experimentales}\label{experimentales}}

\begin{frame}{Características}
\protect\hypertarget{caracteruxedsticas-1}{}
\begin{itemize}
\item
  Condiciones artificiales
\item
  Amplia manipulación
\item
  Necesario diseñar cuidadosamente

  \begin{itemize}
  \tightlist
  \item
    Capturar la mayor variabilidad posible
  \end{itemize}
\end{itemize}
\end{frame}

\begin{frame}{Ejemplos}
\protect\hypertarget{ejemplos-3}{}
\emph{Un equipo de agrónomos seleccionó una muestra de 200 plántulas de
maíz y las dividió en tres grupos. A cada grupo de plántulas las regó
con agua con una concentración diferente de cloruro de sodio para medir
el efecto sobre el crecimiento}
\end{frame}

\begin{frame}{Ejemplos}
\protect\hypertarget{ejemplos-4}{}
\emph{Lee Berger utilizó ranas Littoria caerulea infectadas en
laboratorio con Batrachochytrium dentrobatidis para demostrar que la
quitridiomicosis es la causa de la crisis global de extinción de
anfibios}
\end{frame}

\begin{frame}{Ejemplos}
\protect\hypertarget{ejemplos-5}{}
\emph{Un equipo de científicos ambientales utilizó muestras de
diferentes tipos de suelo, creó diferentes condiciones climátológicas en
cámaras aisladas donde introdujo las muestras y midió los efectos de las
condiciones climatológicas sobre la concentración de nitrógeno y
fósforo}
\end{frame}

\hypertarget{otros-tipos-de-estudios}{%
\section{Otros tipos de estudios}\label{otros-tipos-de-estudios}}

\begin{frame}{Híbridos}
\protect\hypertarget{huxedbridos}{}
\begin{itemize}
\item
  Manipulación de condiciones en el campo

  \begin{itemize}
  \item
    Control menor que en laboratorio
  \item
    Disminuye sesgos observacionales
  \item
    Conclusiones relativamente robustas
  \end{itemize}
\end{itemize}
\end{frame}

\begin{frame}{Factoriales}
\protect\hypertarget{factoriales}{}
\begin{itemize}
\item
  Aplica a observacionales y experimentales
\item
  Estudio del efecto de factores (categóricos generalmente) sobre objeto
  de estudio

  \begin{itemize}
  \tightlist
  \item
    Manipulados o basados en variabilidad del mundo real
  \end{itemize}
\end{itemize}
\end{frame}

\hypertarget{de-cohorte}{%
\subsection{De cohorte}\label{de-cohorte}}

\begin{frame}{De cohorte}
\begin{itemize}
\item
  Aplica a observacionales y experimentales
\item
  Seguimiento a individuos por largos períodos
\item
  Estudios de población
\item
  Comunes en ciencias médicas
\end{itemize}
\end{frame}

\end{document}
