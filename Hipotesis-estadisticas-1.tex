% Options for packages loaded elsewhere
\PassOptionsToPackage{unicode}{hyperref}
\PassOptionsToPackage{hyphens}{url}
%
\documentclass[
  11pt,
  ignorenonframetext,
]{beamer}
\usepackage{pgfpages}
\setbeamertemplate{caption}[numbered]
\setbeamertemplate{caption label separator}{: }
\setbeamercolor{caption name}{fg=normal text.fg}
\beamertemplatenavigationsymbolsempty
% Prevent slide breaks in the middle of a paragraph
\widowpenalties 1 10000
\raggedbottom
\setbeamertemplate{part page}{
  \centering
  \begin{beamercolorbox}[sep=16pt,center]{part title}
    \usebeamerfont{part title}\insertpart\par
  \end{beamercolorbox}
}
\setbeamertemplate{section page}{
  \centering
  \begin{beamercolorbox}[sep=12pt,center]{part title}
    \usebeamerfont{section title}\insertsection\par
  \end{beamercolorbox}
}
\setbeamertemplate{subsection page}{
  \centering
  \begin{beamercolorbox}[sep=8pt,center]{part title}
    \usebeamerfont{subsection title}\insertsubsection\par
  \end{beamercolorbox}
}
\AtBeginPart{
  \frame{\partpage}
}
\AtBeginSection{
  \ifbibliography
  \else
    \frame{\sectionpage}
  \fi
}
\AtBeginSubsection{
  \frame{\subsectionpage}
}
\usepackage{amsmath,amssymb}
\usepackage{lmodern}
\usepackage{iftex}
\ifPDFTeX
  \usepackage[T1]{fontenc}
  \usepackage[utf8]{inputenc}
  \usepackage{textcomp} % provide euro and other symbols
\else % if luatex or xetex
  \usepackage{unicode-math}
  \defaultfontfeatures{Scale=MatchLowercase}
  \defaultfontfeatures[\rmfamily]{Ligatures=TeX,Scale=1}
\fi
\usetheme[]{metropolis}
% Use upquote if available, for straight quotes in verbatim environments
\IfFileExists{upquote.sty}{\usepackage{upquote}}{}
\IfFileExists{microtype.sty}{% use microtype if available
  \usepackage[]{microtype}
  \UseMicrotypeSet[protrusion]{basicmath} % disable protrusion for tt fonts
}{}
\makeatletter
\@ifundefined{KOMAClassName}{% if non-KOMA class
  \IfFileExists{parskip.sty}{%
    \usepackage{parskip}
  }{% else
    \setlength{\parindent}{0pt}
    \setlength{\parskip}{6pt plus 2pt minus 1pt}}
}{% if KOMA class
  \KOMAoptions{parskip=half}}
\makeatother
\usepackage{xcolor}
\newif\ifbibliography
\setlength{\emergencystretch}{3em} % prevent overfull lines
\providecommand{\tightlist}{%
  \setlength{\itemsep}{0pt}\setlength{\parskip}{0pt}}
\setcounter{secnumdepth}{-\maxdimen} % remove section numbering
\ifLuaTeX
  \usepackage{selnolig}  % disable illegal ligatures
\fi
\IfFileExists{bookmark.sty}{\usepackage{bookmark}}{\usepackage{hyperref}}
\IfFileExists{xurl.sty}{\usepackage{xurl}}{} % add URL line breaks if available
\urlstyle{same} % disable monospaced font for URLs
\hypersetup{
  pdftitle={Planteamiento y diseño de la investigación},
  pdfauthor={Gerardo Martín},
  hidelinks,
  pdfcreator={LaTeX via pandoc}}

\title{Planteamiento y diseño de la investigación}
\subtitle{Hipótesis estadísticas}
\author{Gerardo Martín}
\date{2022-06-29}

\begin{document}
\frame{\titlepage}

\begin{frame}{Diferencias con hipótesis científicas}
\protect\hypertarget{diferencias-con-hipuxf3tesis-cientuxedficas}{}
Científicas:

\begin{itemize}
\tightlist
\item
  Aseveración sobre el resultado posible de un experimento
\end{itemize}

Estadísticas:

\begin{itemize}
\tightlist
\item
  Aseveración sobre la probabilidad de que un evento observado ocurra
\end{itemize}
\end{frame}

\begin{frame}{Ejemplos}
\protect\hypertarget{ejemplos}{}
\textbf{Experimento}: Se prueba el rendimiento por hectárea de dos tipos
de pasto, uno nativo y uno introducido, para ganadería extensiva.

Científica:

\begin{itemize}
\tightlist
\item
  Se espera que el pasto nativo tenga mayor rendimiento por hectárea
  debido a que está mejor adaptado a las condiciones ambientales que el
  pasto introducido.
\end{itemize}

Estadística:

\begin{itemize}
\tightlist
\item
  \(H_0\): No hay diferencia entre rendimiento de los pastos
\item
  \(H_1\): Hay diferencias de rendimiento
\end{itemize}
\end{frame}

\begin{frame}{¿Cómo se prueba una hipótesis estadística?}
\protect\hypertarget{cuxf3mo-se-prueba-una-hipuxf3tesis-estaduxedstica}{}
\begin{enumerate}
\item
  Probabilidad de que evento ocurra aleatoriamente
\item
  Se asume probabilidad de corte:
\end{enumerate}

Cuando \(P \leq 0.05\) (la probabilidad de que ocurra aleatoriamente sea
menor a \(0.05\)), se considera que no es aleatorio
\end{frame}

\begin{frame}{Ejemplo}
\protect\hypertarget{ejemplo}{}
\begin{itemize}
\item
  Cuatro hermanxs se sortean diariamente con papelitos el lavado de los
  trastes.
\item
  El más grande organiza la tómbola
\item
  Después de 3 días seguidos sin lavar trastes, los más pequeños
  sospechan que el grande hace trampa
\end{itemize}
\end{frame}

\begin{frame}{Ejemplo}
\protect\hypertarget{ejemplo-1}{}
¿Cómo sabemos si ha hecho trampa o no?

R: Medimos la probabilidad de que no sea seleccionado cada día

\[P_{no} = \frac{3}{4}\]
\end{frame}

\begin{frame}{Ejemplo}
\protect\hypertarget{ejemplo-2}{}
Por lo tanto la probabilidad de no ser seleccionado tres días seguidos
es:

\[P_{no} \times P_{no} \times P_{no} = \frac{3}{4}^3 \approx 0.42\]

¿Es aleatorio?
\end{frame}

\begin{frame}{Ejemplo}
\protect\hypertarget{ejemplo-3}{}
Lxs hermanxs deciden que lo ocurrido es muy probable de observar
aleatoriamente y que el hermano grande no ha hecho trampa.

Continúan de la misma manera por otras dos semanas y e hermano grande
sigue sin lavar los trastes, ¿cuál es la probabilidad de que ello
ocurra?
\end{frame}

\hypertarget{caracteruxedsticas-de-he}{%
\section{\texorpdfstring{Características de
\emph{HE}}{Características de HE}}\label{caracteruxedsticas-de-he}}

\begin{frame}{Poblaciones}
\protect\hypertarget{poblaciones}{}
\begin{itemize}
\item
  Aseveraciones probabilísticas
\item
  \textbf{Siempre} representan poblaciones
\end{itemize}
\end{frame}

\begin{frame}{Hay dos tipos}
\protect\hypertarget{hay-dos-tipos}{}
\begin{enumerate}
\item
  \(H_0\) No se cumple la condición
\item
  \(H_1\) Sí se cumple
\end{enumerate}
\end{frame}

\begin{frame}{Ejercicio}
\protect\hypertarget{ejercicio}{}
Descarga la base de datos y sigue las instrucciones
\end{frame}

\end{document}
