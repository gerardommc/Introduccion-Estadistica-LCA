% Options for packages loaded elsewhere
\PassOptionsToPackage{unicode}{hyperref}
\PassOptionsToPackage{hyphens}{url}
%
\documentclass[
  11pt,
  ignorenonframetext,
]{beamer}
\usepackage{pgfpages}
\setbeamertemplate{caption}[numbered]
\setbeamertemplate{caption label separator}{: }
\setbeamercolor{caption name}{fg=normal text.fg}
\beamertemplatenavigationsymbolsempty
% Prevent slide breaks in the middle of a paragraph
\widowpenalties 1 10000
\raggedbottom
\setbeamertemplate{part page}{
  \centering
  \begin{beamercolorbox}[sep=16pt,center]{part title}
    \usebeamerfont{part title}\insertpart\par
  \end{beamercolorbox}
}
\setbeamertemplate{section page}{
  \centering
  \begin{beamercolorbox}[sep=12pt,center]{part title}
    \usebeamerfont{section title}\insertsection\par
  \end{beamercolorbox}
}
\setbeamertemplate{subsection page}{
  \centering
  \begin{beamercolorbox}[sep=8pt,center]{part title}
    \usebeamerfont{subsection title}\insertsubsection\par
  \end{beamercolorbox}
}
\AtBeginPart{
  \frame{\partpage}
}
\AtBeginSection{
  \ifbibliography
  \else
    \frame{\sectionpage}
  \fi
}
\AtBeginSubsection{
  \frame{\subsectionpage}
}
\usepackage{amsmath,amssymb}
\usepackage{lmodern}
\usepackage{iftex}
\ifPDFTeX
  \usepackage[T1]{fontenc}
  \usepackage[utf8]{inputenc}
  \usepackage{textcomp} % provide euro and other symbols
\else % if luatex or xetex
  \usepackage{unicode-math}
  \defaultfontfeatures{Scale=MatchLowercase}
  \defaultfontfeatures[\rmfamily]{Ligatures=TeX,Scale=1}
\fi
\usetheme[]{metropolis}
% Use upquote if available, for straight quotes in verbatim environments
\IfFileExists{upquote.sty}{\usepackage{upquote}}{}
\IfFileExists{microtype.sty}{% use microtype if available
  \usepackage[]{microtype}
  \UseMicrotypeSet[protrusion]{basicmath} % disable protrusion for tt fonts
}{}
\makeatletter
\@ifundefined{KOMAClassName}{% if non-KOMA class
  \IfFileExists{parskip.sty}{%
    \usepackage{parskip}
  }{% else
    \setlength{\parindent}{0pt}
    \setlength{\parskip}{6pt plus 2pt minus 1pt}}
}{% if KOMA class
  \KOMAoptions{parskip=half}}
\makeatother
\usepackage{xcolor}
\newif\ifbibliography
\setlength{\emergencystretch}{3em} % prevent overfull lines
\providecommand{\tightlist}{%
  \setlength{\itemsep}{0pt}\setlength{\parskip}{0pt}}
\setcounter{secnumdepth}{-\maxdimen} % remove section numbering
\ifLuaTeX
  \usepackage{selnolig}  % disable illegal ligatures
\fi
\IfFileExists{bookmark.sty}{\usepackage{bookmark}}{\usepackage{hyperref}}
\IfFileExists{xurl.sty}{\usepackage{xurl}}{} % add URL line breaks if available
\urlstyle{same} % disable monospaced font for URLs
\hypersetup{
  pdftitle={Cálculo de intervalos de confianza},
  pdfauthor={Gerardo Martín},
  hidelinks,
  pdfcreator={LaTeX via pandoc}}

\title{Cálculo de intervalos de confianza}
\author{Gerardo Martín}
\date{2022-06-29}

\begin{document}
\frame{\titlepage}

\begin{frame}{¿Qué son?}
\protect\hypertarget{quuxe9-son}{}
\begin{itemize}
\item
  Los intervalos de confianza son el error esperado en la estimación de
  la media
\item
  Buena practica calcularlos para conocer la posible variación entre
  muestras
\end{itemize}
\end{frame}

\begin{frame}{¿Cómo se calculan?}
\protect\hypertarget{cuxf3mo-se-calculan}{}
\begin{enumerate}
\tightlist
\item
  Se calcula la media \(\mu\)
\item
  Se calcula la desviación estándar \(\sigma\)
\item
  Se calcula la raíz cuadrada del tamaño de muestra \(\sqrt{n}\)
\item
  Se calcula el error estándar: \(\sigma/\sqrt{n}\)
\end{enumerate}
\end{frame}

\begin{frame}{¿Cómo se representan?}
\protect\hypertarget{cuxf3mo-se-representan}{}
Indican un rango esperado de error en el cáculo de la media, por lo
tanto:

\[\mathrm{IC}_{95} = \bar{x} \pm 1.96 \times \sigma /\sqrt{n}\]
\end{frame}

\begin{frame}{Ejemplo}
\protect\hypertarget{ejemplo}{}
Si tenemos una variable:

\[X = \{ 16.1, 15.6, 14.2, 13.3, 17.7, 14.7, 16.3, 16.0, 11.4, 14.5 \}\]

Su media es:

\[\bar{x} = 14.98\]

Y desviación estándar:

\[\sigma = 1.77\]
\end{frame}

\begin{frame}{Ejemplo}
\protect\hypertarget{ejemplo-1}{}
Los intervalos de confianza al 95\% son:

\[\mathrm{IC}_{95} = 14.98 \pm 1.96 \times 1.77/\sqrt{10}\]

Por lo tanto:

\[\mathrm{IC}_{95} = 14.98 \pm 1.1\] Por lo que en el 95\% de los casos
que estimemo la media de la población si la muestra \(X\) es
representativa, la media quedará en el intervalo

\[\{16.1, 13.88\}\]
\end{frame}

\end{document}
